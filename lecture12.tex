% Created 2023-11-11 Sat 17:22
% Intended LaTeX compiler: pdflatex
\documentclass[11pt]{article}
\usepackage[utf8]{inputenc}
\usepackage[T1]{fontenc}
\usepackage{graphicx}
\usepackage{longtable}
\usepackage{wrapfig}
\usepackage{rotating}
\usepackage[normalem]{ulem}
\usepackage{amsmath}
\usepackage{amssymb}
\usepackage{capt-of}
\usepackage{hyperref}
\author{Aidan  Robinson}
\date{\today}
\title{Lecture12}
\hypersetup{
 pdfauthor={Aidan  Robinson},
 pdftitle={Lecture12},
 pdfkeywords={},
 pdfsubject={},
 pdfcreator={Emacs 29.1 (Org mode 9.7)}, 
 pdflang={English}}
\begin{document}

\maketitle
\tableofcontents

\section{File system design \& implementation}
\label{sec:orgbfe35e5}
\subsection{Motivation}
\label{sec:org69fdc74}
\begin{itemize}
\item performance
\begin{itemize}
\item latency
\item throughput
\item space (Secondary storage)
\end{itemize}
\item Safety/robustness
\begin{itemize}
\item crashes
\item Bad devices
\end{itemize}
\end{itemize}
\section{Unix style}
\label{sec:orgf5abcc0}
\begin{itemize}
\item Hybrid data structure:
\begin{itemize}
\item L1/L2 RAM vs Secondary storage
\end{itemize}
\item NUMA Architecture: Non uniform memeory access
\begin{itemize}
\item makes a big difference in speed of access
\end{itemize}
\end{itemize}
\subsection{Inodes}
\label{sec:org0a403c8}
\begin{itemize}
\item[{$\boxtimes$}] Benefit: very flexible (see last lecture)
\item Disadvantage: half to go to the flashdrive twice (inode, then file associated with inode)
\item Inode structure:
\begin{itemize}
\item files meta data
\begin{itemize}
\item perm, owner, timestamp, link count, inode file
\end{itemize}
\item Inodes do NOT tell you the files name
\begin{itemize}
\item because of hard links:
\item 2 hard links to same inode, what name to use?
\end{itemize}
\end{itemize}
\item Addressses of file data (blocks)
\item Problems:
\begin{itemize}
\item large files suffer from bad seek times
\item treating 2 files linked to same inode causes bad behavior
\item deleting files is more complicated
\begin{itemize}
\item Solution: referene/link count INodes \& the disk space, count drops to 0
\end{itemize}
\end{itemize}
\item Link counts:
\begin{itemize}
\item if count drops to 0, AND no processes have the file open
\begin{itemize}
\item uses hybrid storage to enforce:
\begin{itemize}
\item Link count is part of permanent storage,
\item \# of processes with file open on ram
\end{itemize}
\end{itemize}
\item Link count problem: LEAKS:
\begin{itemize}
\item if link count 0, but computer crashes while reading,
\item Solution: `fsck` file system check @ /dev/sdc
\item looks for bad stuff (i.e. inodes with link count 0)
\end{itemize}
\end{itemize}
\end{itemize}
\subsection{Renames}
\label{sec:org17e9a6f}
\begin{itemize}
\item `mv a b`: read 1 block; write 1 block (the name)
\item `mv d/a e/b`: renaming in directories is tough
\begin{itemize}
\item increment d/a link count by 1 (write to inode)
\item create an entry b in directory e (write to data block of directory e)
\item remove entry a in d (write)
\item Decrementing inode 19's link count (write)
\end{itemize}
\item `ln d/a e/b`:
\begin{itemize}
\item increment d/a link count by 1 (write to inode)
\item create an entry b in directory e (write to data block of directory e)
\item benefit: SAFER to implement link and unlink
\end{itemize}
\end{itemize}
\subsection{Directory entries for varyting length file name components}
\label{sec:org1a712ba}
\begin{itemize}
\item components: name between the ``/''
\item 255 byte limit for file name components: static
\end{itemize}
\subsection{Limits \& directory entry}
\label{sec:org4456ba9}
\begin{itemize}
\item ext2/3/4 \& 4BSD File system: different directory entry structure
\item 32-bit inode \#
\item 16-bit directory entry length
\begin{itemize}
\item lets you delete quickly by adding directory entry length to previous entry's
\begin{itemize}
\item We don't squash the files together!
\end{itemize}
\item[{$\boxtimes$}] draw this out!
\item 1 block rewriting!
\item directory entry length can be MUCH bigger than name length
\end{itemize}
\item 8-bit name length
\item 8 bit file type
\begin{itemize}
\item a copy of what is already in the inode
\item why cache file type in the directory entry? for performance
\begin{itemize}
\item for recursive calls, checking if an entry is a file is faster (no system call needed)
\end{itemize}
\item Why not cache the size: because it is changing/mutable (file type is not!)
\end{itemize}
\item file name component: component entry: N size (specified by 8-bit name length)
\item Entries do not cross block boundaries
\end{itemize}
\subsection{Cyclic inodes}
\label{sec:org07feb90}
\begin{verbatim}
mkdir a
ln a a/b

\end{verbatim}
\begin{itemize}
\item problem: `find a -type f`
\item cannot hardlink to directories
\begin{itemize}
\item otherwise, a, a/b, a/b/b would all be valid
\item Want to prevent cycles
\end{itemize}
\item for `link()` system call, fail if first argument is a directory
\end{itemize}
\subsection{. problem:}
\label{sec:orgf0476f9}
\begin{itemize}
\item `opendir('.')` cannot be done, because inode doens't have name
\item inode number of current working directory is stored in process table entry
\item `open(``a/b/c'')`: will look component by component for inode numbers
\begin{itemize}
\item if absolute path, start with root directory inode \#
\end{itemize}
\item Chdir system call will change the pwd column in process table entry
\begin{itemize}
\item shell's directory is not changed, only the cd.c process changes working directory
\begin{verbatim}
  // Incorrect code!
    int main (int argc, char ** argv){
        if(chdir(argv[1]) < 0){
          perror("chdir");
          return 1;
        }
        return 0;
    }



\end{verbatim}
\end{itemize}
\end{itemize}
\subsubsection{Chroot shenanigans:}
\label{sec:org3e8b891}
\begin{itemize}
\item chroot(char* new\textsubscript{root});
\begin{itemize}
\item no way to change it back, /.. = /
\item only usable by root: otherwise you could do shananigans
\end{itemize}
\item chroot could let create your own /etc/password file, and run sudo commands (dangerous!)
\end{itemize}
\subsection{Symbolic links}
\label{sec:orgd062b92}
\begin{itemize}
\item implemented as a regular file, just specify that it is a symbolic link
\item Symlinks are relative links!
\item Meta data is same as regular inode, except file type is a symlink (\& less data blocks)
\item symlink file type is treated as a reference,
\begin{itemize}
\item replaces contents of name of sym link with the contents of the sym link
\item Symlinks are expanded at `namei` time: expanded by the kernel
\begin{itemize}
\item namei: traversal
\end{itemize}
\end{itemize}
\end{itemize}
\subsubsection{Hardlinks with symbolic links}
\label{sec:orgf9a0a35}
\begin{itemize}
\item can create a hardlink to a symbolic link
\end{itemize}
\begin{verbatim}
ln -s a d/b
ln d/b e/f

cat d/b # same as cat d/a
cat e/f # same cat e/f -> cat e/a
# directory d and directory e are not
# sym links are resolved inside the current directory?
\end{verbatim}
\begin{itemize}
\item \textbf{\textbf{A realtive symbolic link is interpreted in the context of the directory that contains it}}
\item Purpose: can copy a tree easily
\end{itemize}
\subsubsection{Symlink problems:}
\label{sec:org28d689e}
\begin{itemize}
\item dangling symlink
\item cyclical symlinks (a linked to b, b linked to a)
\begin{itemize}
\item namei has a symlink limit
\item errno = ELOOP
\end{itemize}
\item can't edit symlink directly, because `open` will expand
\item some system calls don't follow sym links
\begin{itemize}
\item ls -l foo
\end{itemize}
\end{itemize}
\subsection{/dev/null: other file types}
\label{sec:org83970f0}
\begin{itemize}
\item c type file: kernel module designed to mimic that file
\begin{itemize}
\item /dev/null pretends to be a file, returns end of file immediatley
\end{itemize}
\item mknod /dev/null c 1 3
\begin{itemize}
\item create a character special file
\item major device number \& minor device number: specify what part of the kernel to invoke when the file is called
\end{itemize}
\item[{$\square$}] block special file
\item pipes: mkfifo /tmp/f
\begin{itemize}
\item creates a pipe in the kernel,
\item Creates entry in file system pointing to the pipe
\item cat /tmp/f: reads the pipe
\item writing to pipe: echo abc > /tmp/f
\item easy way to allow communication w/o saving a file to disk or a system call
\end{itemize}
\end{itemize}
\subsection{One more thing}
\label{sec:org2a6ae83}
\begin{itemize}
\item seperate file system per each devices
\begin{itemize}
\item makes changing out devices easier
\end{itemize}
\item Need to number devices
\begin{itemize}
\item accessing files needs file number \& device number
\item linux must keep track of new file system
\end{itemize}
\item Mounted file systems
\begin{itemize}
\item table keeping track of your mounted systems
\end{itemize}
\end{itemize}
\end{document}